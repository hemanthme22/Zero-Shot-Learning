% Hide page numbers for this page.
\thispagestyle{empty}
\newgeometry{top=1.75in, left=1.5in, right=1in, bottom=1in}
\begin{center}
    EXPLORING THE LIMITS OF ZERO-SHOT LEARNING - HOW LOW CAN YOU GO ?
    
    by
    
    HEMANTH DANDU
    
    (Under the Direction of Suchendra Bhandarkar)
    
    ABSTRACT
\end{center}
Zero-shot learning aims to classify input data into categories with zero training examples. The classification is performed by inferring unseen categories using visual data of seen categories and their relationships with unseen categories. Relationships are determined by using auxiliary data pertaining to categories such as attributes and semantics. Standard zero-shot learning techniques use a large number of seen categories to predict very few unseen categories while maintaining unified data splits and evaluation metrics. This has enabled the research community to advance notably towards formulating a standard benchmark zero-shot learning algorithm. However, the most substantial impact of zero-shot learning lies in enabling the prediction of a large number of unseen categories from very few seen categories within a specific domain. This permits the collection of training data for only a few previously seen categories, thereby mitigating the training data collection process significantly. In this thesis, we focus on the difficult problem of predicting a large number of unseen object categories from very few previously seen categories. We propose a framework that enables us to examine the limits of inferring several unseen object categories from very few previously seen object categories, i.e., the limits of zero-shot learning. In particular, we examine the functional dependence of the classification accuracy of unseen object classes on the number of previously seen classes. We also determine the minimum number of previously seen classes required to achieve pre-specified classification accuracy for the unseen classes on three standard zero-shot learning data sets, i.e., AWA2, CUB and SUN. Additionally, we compare the proposed framework with a prominent zero-shot learning technique on the aforementioned data sets and find that we achieve 21\% higher accuracy on the AWA2 data set, 6\% higher accuracy on the CUB data set, and comparable performance on the SUN data set while providing valuable insights into the unseen class inference process.

%\vspace{2\baselineskip}
\noindent
INDEX WORDS: machine learning, image classification, zero-shot learning, transfer learning

\newpage